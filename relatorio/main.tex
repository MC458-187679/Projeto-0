\documentclass[a4paper, 14pt]{extarticle}

\usepackage[portuguese]{babel}
\usepackage[utf8]{inputenc}
\usepackage[T1]{fontenc}
\usepackage[margin=2.2cm]{geometry}

\usepackage{float, pgf, caption, subcaption}

% \makeatletter

\usepackage{nameref, titlesec}
\usepackage[hidelinks]{hyperref}


% formatação das seções
\titleformat{\section}[runin]
    {\titlerule{}\vspace{1ex}\normalfont\Large\bfseries}{}{1em}{}[.]
\titleformat{\subsection}[runin]
    {\normalfont\large\bfseries}{}{1em}{}[)]

% marcadores de nome na seção
\newcommand{\cur@section}{ERRO}
\newcommand{\cur@subsection}{ERRO}

% seções que atualizam esses marcadores
\let\old@section\section
\renewcommand{\section}[1]{%
    \renewcommand{\cur@section}{#1}%
    \renewcommand{\cur@subsection}{}%
    \old@section{#1}%
}
\renewcommand{\appendix}[1]{%
    \renewcommand{\cur@section}{#1}%
    \renewcommand{\cur@subsection}{}%
    \old@section{Apêndice #1}%
}
\let\old@subsection\subsection
\renewcommand{\subsection}[1]{%
    \renewcommand{\cur@subsection}{.\uppercase{#1}}%
    \old@subsection{#1}%
}

% nome com seção e subseção
\newcommand{\cur@itemname}{\cur@section\cur@subsection}

% comandos de referenciar
\newcommand{\new@ref}[3]{
    \newcommand{#1}[2][]{%
        \ifstrempty{##1}{%
            \hyperref[{##2}]{#2{##2}#3}%
        }{%
            \hyperref[{##1}]{#2{##2}#3}%
        }%
    }
}
\new@ref{\boldref}{\textbf}{}
% \new@ref{\thmref}{\bfseries\ref*}{}
\newcommand{\thmref}[2][]{\hyperref[#2]{\bfseries{}#1\ref*{#2}}}
\new@ref{\itemref}{\bfseries\nameref*}{)}
\new@ref{\qref}{\bfseries\nameref*}{.}

% linha final da página ou seção
\newcommand{\docline}[1][\\]{%
    #1\noindent\rule{\textwidth}{0.4pt}%
    \pagebreak%
}

% pula uma linha
\def\skipline{\vskip\baselineskip}

% linha horizontal menor
\def\itemsep{
    \noindent\hfil\rule{0.5\textwidth}{.2pt}\hfil
    \vskip1em
}

\makeatother

% \makeatletter

\usepackage{amsthm, amsmath, amssymb, bm, mathtools}
\usepackage{enumitem, etoolbox, xpatch}


% tira o ponto do nome do teorema
\AtBeginDocument{\xpatchcmd{\@thm}{\thm@headpunct{.}}{\thm@headpunct{}}{}{}}

% teoremas mais simples
% \newtheorem*{ntheorem}{Teorema.}
\newtheorem*{intheorem}{Teorema Incorreto.}
\newtheorem*{ptheorem}{Teorema?}


% faz label com nome quando não está vazio
\newcommand{\set@label}[2]{
    \ifstrempty{#1}{}{%
        \begingroup%
            \edef\@currentlabel{#2}%
            \label{#1}%
        \endgroup%
    }
}

% prova por casos
\newcounter{cases@cnt}
\newtheoremstyle{case}{0.2em}{0.2em}{}{\parindent}{\itshape}{.}{0.8em}{\thmname{#1}\thmnumber{ #2}\thmnote{: ~#3~}}
\theoremstyle{case}
\newtheorem{case}[cases@cnt]{Caso}
\def\resetCasos{\setcounter{cases@cnt}{0}}

% ambiente base dos teoremas
\newtheoremstyle{theorem@sty}{0.2em}{0.2em}{\itshape}{}{\bfseries}{:}{0.8em}{\thmname{#1}\thmnote{ #3}\thmnumber{.#2}}
\theoremstyle{theorem@sty}
\newtheorem*{theorem@thm}{Teorema}

% contador dos teoremas
\newcounter{theorem@cnt}[subsection]

% ambientes dos teoremas
\newenvironment{theorem}[1][]{%
    \resetCasos%
    \refstepcounter{theorem@cnt}%
    \edef\tmp@val{\thetheorem@cnt}%
    \begin{theorem@thm}[\cur@itemname.\tmp@val]%
        \set@label{#1}{\cur@itemname.\tmp@val}%
}{%
    \end{theorem@thm}%
}
\newenvironment{theorem*}[1][]{%
    \resetCasos%
    \begin{theorem@thm}[\cur@itemname]%
        \set@label{#1}{\cur@itemname}%
}{%
    \end{theorem@thm}%
}

% teorema nomeado
\newenvironment{ntheorem}[2][]{%
    \resetCasos%
    \begin{theorem@thm}[#2]%
        \set@label{#1}{#1}%
}{%
    \end{theorem@thm}%
}

% ambiente para lemas
\newtheorem*{theorem@lemma}{Lema}
\newenvironment{lemma}[1][]{%
    \resetCasos%
    \refstepcounter{theorem@cnt}%
    \edef\tmp@val{\thetheorem@cnt}%
    \begin{theorem@lemma}[\cur@itemname.\tmp@val]%
        \set@label{#1}{\cur@itemname.\tmp@val}%
}{%
    \end{theorem@lemma}%
}

% comentários
\newenvironment{comments}[1][Comentários]{%
    \begin{proof}[#1]%
        \edef\qedsymbol{}%
}{%
    \end{proof}%
}

% objetivo do bloco da prova
\newcommand{\objetivo}[1]{$\left(\text{\textit{#1}}\right)$}

\makeatother

\makeatletter

\usepackage{amsthm, amsmath, amssymb, bm, mathtools}
\usepackage{enumitem, etoolbox, xpatch}
% \usepackage[mathcal]{euscript}
% \usepackage[scr]{rsfso}
\usepackage{mathptmx}
\usepackage{relsize, centernot, tikz, xcolor}


%%%% QED symbols %%%%
\def\qed@open{\ensuremath{\square}}
\def\qed@open@small{\ensuremath{\mathsmaller\qed@open}}

\def\qed@fill{\ensuremath{\blacksquare}}
\def\qed@fill@small{\ensuremath{\mathsmaller\qed@fill}}

\definecolor{qed@gray}{gray}{0.8}
\def\qed@gray{\ensuremath{\color{qed@gray}\blacksquare}}
\def\qed@gray@small{\ensuremath{\color{qed@gray}\mathsmaller\blacksquare}}

\def\qed@both@cmd#1#2{\begin{tikzpicture}[baseline=#2]
    \draw (0,0) [fill=qed@gray] rectangle (#1,#1);
\end{tikzpicture}}
\def\qed@both{\qed@both@cmd{0.6em}{0.2ex}}
\def\qed@both@small{\qed@both@cmd{1ex}{0ex}}

\def\showAllQED{
    \qed@open ~ \qed@open@small \\
    \qed@fill ~ \qed@fill@small \\
    \qed@gray ~ \qed@gray@small \\
    \qed@both ~ \qed@both@small
}

%% escoha do QED %%
\renewcommand{\qedsymbol}{\qed@fill@small}

% marcadores de prova
\newcommand{\direto}[1][~]{\ensuremath{(\rightarrow)}#1}
\newcommand{\inverso}[1][~]{\ensuremath{(\leftarrow)}#1}

% fontes
% conjunto potencia
\DeclareSymbolFont{boondox}{U}{BOONDOX-cal}{m}{n}
\DeclareMathSymbol{\pow}{\mathalpha}{boondox}{"50}

% somatorio
\DeclareSymbolFont{matext}{OMX}{cmex}{m}{n}
\DeclareMathSymbol{\sum@d}{\mathop}{matext}{"58}
\DeclareMathSymbol{\sum@t}{\mathop}{matext}{"50}
\undef\sum
\DeclareMathOperator*{\sum}{\mathchoice{\sum@d}{\sum@t}{\sum@t}{\sum@t}}
\DeclareMathOperator*{\bigsum}{\mathlarger{\mathlarger{\sum@d}}}

% phi computer modern
\DeclareMathAlphabet{\gk@mf}{OT1}{cmr}{m}{n}
\let\old@Phi\Phi
\def\Phi{\gk@mf{\old@Phi}}

% união elem por elem
\DeclareMathOperator{\wcup}{\mathaccent\cdot\cup}

% familia de conjuntos
\undef\fam
\DeclareMathAlphabet{\fam}{OMS}{cmsy}{m}{n}

% alguns símbolos
\undef\natural
\DeclareMathOperator{\real}{\mathbb{R}}
\DeclareMathOperator{\natural}{\mathbb{N}}
\DeclareMathOperator{\integer}{\mathbb{Z}}
\DeclareMathOperator{\complex}{\mathbb{C}}
\DeclareMathOperator{\rational}{\mathbb{Q}}
\def\symdif{\mathrel{\triangle}}
\def\midd{\;\middle|\;}
% \def\pow{\mathcal{P}}

% operações com mais espaçamento
\def\cupp{\mathbin{\,\cup\,}}
\def\capp{\mathbin{\,\cap\,}}

% alguns operadores
\DeclareMathOperator{\Dom}{Dom}
\DeclareMathOperator{\Img}{Im}

% marcadores de operadores
\def\inv{^{-1}}
\def\rel#1{\use@invr{\!\mathrel{#1}\!}}
\def\nrel#1{\use@invr{\!\centernot{#1}\!}}
\def\dmod#1{\ (\mathrm{mod}\ #1)}
\def\cgc#1{{\textnormal{[}#1\textnormal{]}}}
\def\cgp#1{{\textnormal{(}#1\textnormal{)}}}

% marcador com inverso reduzido
\def\use@invr#1{%
    \begingroup%
        \edef\inv{\inv\!}%
        #1%
    \endgroup%
}

% delimiters
\def\abs#1{{\lvert\,#1\,\rvert}}
% \DeclarePairedDelimiter{\abs}{\lvert}{\:\rvert}

\makeatother


% math display skip
\newcommand{\reducemathskip}[1][0.5em]{%
    \setlength{\abovedisplayskip}{1pt}%
    \setlength{\belowdisplayskip}{#1}%
    \setlength{\abovedisplayshortskip}{#1}%
    \setlength{\belowdisplayshortskip}{#1}%
}

% url linking problems
\def\url#1{\href{#1}{\texttt{#1}}}
% vermelho
\def\red#1{\textcolor{red}{#1}}

\def\lmref#1{\thmref[lema ]{#1}}

\usepackage{xparse}

\newtheorem*{hypothesis}{Hipótese}
\newtheorem*{hypothesisf}{Hipótese Fortalecida}

\NewDocumentCommand{\seq}{ s m O{n} O{\in\natural} }
    {\IfBooleanTF{#1}
        {\ensuremath{\left({#2}_{#3}\right)}}
        {\ensuremath{\left({#2}_{#3}\right)_{{#3}{#4}}}}}


\title{\vspace{-2.5cm}Projeto de Algoritmo com Implementação nº 0 \\ \normalsize MC458 - 2s2020}
\author{Tiago de Paula Alves \\ \small 187679}
\date{}

\begin{document}
\maketitle

\section{Implementação}

    O algoritmo de cálculo de estabilidade foi feito considerando o móbile como uma árvore especializada. No caso, cada móbile tem dois objetos $O_e$ e $O_d$, com distâncias $D_e$ e $D_d$ do sustentáculo. Cada objeto $O$ pode ser um outro móbile ou um peso simples, sendo o peso $P_O$ conhecido apenas de objetos simples. Dessa forma, um móbile $M$ funciona como uma árvore não-vazia, sendo cada submódule um nó interno da árvore e objetos simples são as suas folhas.

    Assim, cada peso simples $S$ tem uma medida de peso $P_S$ fixa, recebida da entrada padrão. Os móbiles $M$, no entanto, têm peso dependente de seus objetos $E$ e $D$. Então, podemos definir a função peso $P: \text{Objeto} \to \real$, que calcula o peso total de um objeto $O$, como:
    \[
        P(O) = \begin{cases}
            P_O & \text{se $O$ é um peso simples} \\
            P(E) + P(D) & \text{se $O$ é um móbile com objetos $E$ e $D$}
        \end{cases}
    \]

    Com isso, um móbile $M$ com objetos $E$ e $D$, de pesos $P_E$ e $P_D$ e às distâncias $D_E$ e $D_D$, está em equilíbrio se seus submóbiles estão em equilíbrio e se $P_E \cdot D_E = P_D \cdot D_D$. Expandindo para objetos em geral, podemos tratar um peso simples como sempre em equilíbrio, de forma que $M$ estará em equilíbrio se e somente se $E$ está em equilíbrio, $D$ está em equilíbrio e $P(E) \cdot D_E = P(D) \cdot D_D$.

\section{Algoritmo}

    O algoritmo pode ser deduzido a partir de uma indução forte, como comumente acontece para árvores binárias. Nesse caso, a primeira hipótese de indução seria:

    \begin{hypothesis}
        Dado um móbile com $n$ objetos, conseguimos vericar se ele está em equilíbrio.
    \end{hypothesis}

    No entanto, esse tratamento funciona para objetos com peso definidos ou já conhecidos. Então, podemos fortalecer a hipótese com o cálculo da função peso $P$ definida anteriormente. Além disso, podemos usar a definição de equilíbrio que abrange os dois tipos de objetos, não só móbiles, fazendo uma hipótese que encobre os dois casos.

    \begin{hypothesisf}
        Dado um objeto $O$ com $n$ subobjetos, podemos calcular seu peso $P(O)$ e vericar se ele está em equilíbrio.
    \end{hypothesisf}

\section{Alg}

    \begin{proof}
        Suponha um inteiro $n \geq 1$ tal que para todo objeto $O$ composto de $1 \leq k < n$ massas podemos descobrir seu peso total $P(O)$ e se ele está em equilíbrio. Suponha ainda um objeto $O$ com $n$ subobjetos.

        ~

        Caso 1: $O$ é um peso simples, ou seja, $n = 1$. Logo, o peso de é conhecido, então seja $P_O$ esse peso. Portanto, $P(O) = P_O$ e, por definição, $O$ está em equilíbrio.

        ~

        Caso 2: $O$ é um móbile. Seja $E$ o objeto peso na ponta esquerda de $O$, à uma distância $D_E$ do sustentáculo, e $D$ o objeto na direita, com distância $D_D$.

        Note que $D$ deve ter pelo $k_D \geq 1$ massas, que é no caso em que $D$ é um peso simples. Portanto, $E$ terá $1 \leq k_E = n - k_D \leq n - 1 < n$ massas. Logo, pela hipótese indutiva, podemos calcular $P(E)$ e sabemos se $E$ está em equilíbrio. Da mesma forma, $1 \leq k_D < n$. Então, podemos conseguir $P(D)$ e saber se $D$ está em equilíbrio.

        Assim, se $E$ e $D$ estão em equilíbrio e $P(E) * D_E = P(D) * D_D$, então $O$ também está em equilíbrio. Caso alguma dessas condições não seja atendida, $O$ não está em equilíbrio. Portanto, teremos que $P(O) = P(E) + P(D)$ e podemos dizer se $O$ está em equilíbrio.
    \end{proof}


    % \begin{proof}
    %     Suponha um inteiro $n \geq 2$ tal que para todo mobile $M$ composto de $2 \leq k < n$ pesos simples podemos descobrir seu peso $P(M)$ e se ele está em equilíbrio. Suponha ainda um móbile $M$ com $n$ pesos.

    %     Caso 1: $M$ é um móbilo simples. Logo, temos dois pesos simples $e$ e $d$, com pesos $P_e$ e $P_d$ e
    % \end{proof}

\end{document}
